\documentclass{article}

\usepackage[tmargin=0.5in,bmargin=0.25in]{geometry}
\usepackage{amsmath, amssymb, amsthm}
\usepackage{enumitem}
\usepackage{tikz}
\usepackage{multicol}

\title{\vspace{-5ex}MATH 341 Study Guide \\ Midterm 1}
\author{Isaac Boaz}

\begin{document}
\maketitle

\section*{Definitions and Laws}
\begin{enumerate}[noitemsep]
    \item \(A\) and \(B\) are \textbf{mutually exclusive} or \textbf{disjoint} if \(A \cap B = \varnothing\).
    \item \(A_1, A_2, \dots, A_k\) are \textbf{exhaustive events} if \(A_1 \cup A_2 \cup \dots \cup A_k = S\).
    \item Commutative laws
          \begin{itemize}
              \item \(A \cup B = B \cup A\)
              \item \(A \cap B = B \cap A\)
          \end{itemize}
    \item Associative laws
          \begin{itemize}
              \item \(A \cup (B \cup C) = (A \cup B) \cup C\)
              \item \(A \cap (B \cap C) = (A \cap B) \cap C\)
          \end{itemize}
    \item Distributive laws
          \begin{itemize}
              \item \(A \cap (B \cup C) = (A \cap B) \cup (A \cap C)\)
              \item \(A \cup (B \cap C) = (A \cup B) \cap (A \cup C)\)
          \end{itemize}
    \item De Morgan's laws
          \begin{itemize}
              \item \(\overline{(A \cup B)} = \overline{A} \cap \overline{B}\)
              \item \(\overline{(A \cap B)} = \overline{A} \cup \overline{B}\)
          \end{itemize}
\end{enumerate}

\section*{Axioms}
\begin{enumerate}
    \item For any event \(A\) in \(S,\ 0 \leq P(A) \leq1\).
    \item P(S) = 1.
    \item For any sequence of mutually exclusive (disjoint) events \(A_1, A_2, A_3, \dots \text{in} S (\text{i.e., } A_i \cap A_j = \varnothing \text{ whenever } i \neq j\)), then
          \begin{equation*}
              P(A_1 \cup A_2 \cup A_3 \dots) = \sum_{i=1}^{=}P(A_i).
          \end{equation*}
\end{enumerate}

These axioms imply that
\begin{itemize}
    \item \(P(\varnothing) = 0.\)
    \item \(P(A \cup B) = P(A) + P(B)\) when \(A\) and \(B\) are mutually exclusive.
\end{itemize}

\section*{Permutations and Combinations}
\begin{center}
    \begin{tabular}{|c|c|c|}
        \hline
        \rule{0pt}{3ex} Select \(r\) from \(n\) & Order matters                    & Order does not matter                              \\[1ex]
        \hline
        \rule{0pt}{3ex} Without replacement     & \({}_nP_r  = \frac{n!}{(n-r)!}\) & \({}_nC_r = \binom{n}{r} = \frac{n!}{r!(n - r)!}\) \\[1ex]
        \hline
        \rule{0pt}{3ex} With replacement        & \(n^r\)                          & \(\binom{n+r-1}{r}\)                               \\[1ex]
        \hline
    \end{tabular}
\end{center}

\pagebreak

\section*{Conditional Probability}
The \textbf{conditional probability} of \(A\) given \(B\) has occured is
\begin{multicols}{2}{\parbox[c][2em]{\hsize}{
            \begin{equation*}
                P(A \mid B) = \frac{P(A \cap B)}{P(B)}
            \end{equation*}
        }}
    \columnbreak
    \begin{equation*}
        P(A \cap B) = P(A \mid B)P(B).
    \end{equation*}
\end{multicols}

\section*{Bayes' Theorem}
\begin{equation*}
    P(A \mid B) = \frac{P(B \mid A)P(A)}{P(B)}.
\end{equation*}

\end{document}