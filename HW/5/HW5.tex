\documentclass{article}

\usepackage[tmargin=0.5in,bmargin=0.25in]{geometry}
\usepackage{amsmath, amssymb, amsthm}
\usepackage{enumitem}

\title{Math 341 Homework 5}
\author{Isaac Boaz}

\renewcommand{\arraystretch}{1.2}

\begin{document}

\maketitle

\section*{Problem 2}
\begin{enumerate}[label=\alph*)]
    \item Can be modeled as the Binomial Distribution \(x \sim b(8, 0.001)\) where \(n = 8,\ p = 0.001\)
    \item No, the cars differ, track conditions may be different, different drivers
          Additionally, one race being faster may motivate a driver and impact them mentally to do better/worse
    \item No, the poisson distribution requires \(n\) to be large, \(p\) to be small, and that \(np\) to be fixed. In this case, \(n\) is very small.
    \item \(Y \approx X\) in distribution.
          \begin{align*}
              Y & \sim Poisson(\lambda). \\
              \frac{e^{-Y}Y^x}{x!}
          \end{align*}
    \item \(P(Y = 0),\ P(X = 0)\) % These should be very close
    \item Q1, Q2
          Referencing d for one session, doubling these sessions keeps the poisson distribution.
          Let W be the \# of drivers.
          \begin{align*}
              W \sim Poisson(2\lambda) \\
              f
              %       P(W = 0) & = P(X = 0, Y = 0)                                                     \\
              %                & = P(X = 0)P(Y = 0)                                                    \\
              %                & = \frac{e^{-\lambda} \lambda^0}{0!} \frac{e^{-\lambda} \lambda^0}{0!} \\
              %                & = \frac{e^{-2\lambda} \lambda^0}{0!}                                  \\
              %                & = \frac{e^{-2\lambda}}{1}                                             \\
              %                & = e^{-2\lambda}
          \end{align*}
\end{enumerate}

\section*{Problem 3}
\begin{enumerate}[label=\alph*)]
    \item Assume \(p = 0.1,\ n = 24\). Poisson with parameter \(np\) = 2.4
    \item Binomial takes \(\approx 2 \times\) longer at \(2.34\) for poisson vs \(4.72\) for binomial.
\end{enumerate}

\section*{Problem 14}
Suppose \(X\) is uniformly distributed in \([a, b]\).
\begin{enumerate}[label=\alph*)]
    \item Let \(a \leq c \leq x \leq b\) with \(a \leq c < b\). Compute \(P(X \leq x \mid X \geq c)\).
          \begin{align*}
              X                         & \sim U(a, b)                                    \\
              P(X \leq x \mid X \geq c) & = \frac{P(X \leq x \cap X \geq c)}{P(X \geq c)} \\
                                        & = \frac{P(c\leq X \leq x)}{P(X \geq c)}         \\
                                        & = \frac{\int_{c}^{x}f(t)dt}{\int_{c}^{b}f(t)dt} \\
                                        & = \frac{x - c}{b - c}
          \end{align*}
    \item Explain \\
          The lower bound is not \(a\), we are bounding it to c. Since everything else is the same, we can conclude that the distribution is uniform in \([c, b]\).
    \item Blah blah blah \\
          Let \(V \sim U(a, c)\), show \(E[X] = wE[V] + (1-w)E[Y]\) where \(w = \frac{c-a}{b-a}\)
          \begin{align*}
              E[Y] = \frac{b+c}{2} \\
              E[V] = \frac{a+c}{2} \\
              % Simplify right hand side, w E[V] + (1-w) E[Y]
          \end{align*}
\end{enumerate}

\end{document}