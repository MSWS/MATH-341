\documentclass{article}

\usepackage[tmargin=0.5in,bmargin=0.25in]{geometry}
\usepackage{amsmath, amssymb, amsthm}
\usepackage{enumitem}

\title{\vspace{-4ex}Math 341 Homework 6}
\author{Isaac Boaz}

\renewcommand{\arraystretch}{1.2}

\begin{document}

\maketitle

\section*{Problem 2}

Let \(X\) be a random variable representing the number of shooting stars per hour. Assume \(X\) is Poisson distributed with \(E[x] = \lambda\) (i.e: \(X \sim Poisson(\lambda)\)).

\begin{enumerate}[label=\alph*)]
    \item Instead of a 1-hour interval, consider an interval of \(t\) hours for some \(t > 0\). If \(Y\) denotes the number of stars in \(t\) hours, what is its distribution and parameter value?
          \begin{align*}
              Y \sim Poisson(t\lambda)
          \end{align*}
    \item Calculate the probability that no shooting stars are observed in \(t\) hours (i.e \(P(Y = 0)\)).
          \begin{align*}
              P(Y = y) = \frac{e^{-t\lambda} \cdot (t\lambda)^y}{y!}
          \end{align*}
          \begin{align*}
              P(Y = 0) & = \frac{e^{-t\lambda} \cdot (t\lambda)^0}{0!} \\
                       & = \frac{e^{-t\lambda}}{1}                     \\
                       & = e^{-t\lambda}                               \\
          \end{align*}
    \item Suppose we measure the time until the first shooting star. Let \(T\) denote this time. Explain why the event \(A = \{T > t\}\) is equivalent to the event \(B = \{Y = 0\}\)
          \begin{description}
              \item Event \(B\) represents the event that no shooting stars are observed in \(t\) hours.
              \item Event \(A\) represents the event that the first shooting star is observed after \(t\) hours.
              \item Since event \(A\) is defined as \(T > t\), we know that at least \(t\) hours have passed before the first shooting star is observed.
          \end{description}
    \item Using (c), compute the cdf of \(T\). Then, state the distribution of \(T\) and its parameter value.
          \begin{align*}
              A = B    & \implies P(A) = P(B) \\
              P(T > t) & =  P(Y = 0)          \\
                       & = e^{-t\lambda}      \\
          \end{align*}
          \begin{align*}
              \text{CDF of T: } F(t) =  P(T \leq t) \\
              = 1 - e^{-t\lambda}
          \end{align*}
          Exponential distribution with parameter \(\lambda\).
\end{enumerate}

\pagebreak

\section*{Problem 7}
Let \(X\) be a random variable for the percentage of Associate's degree holders. \\
Estimated mean = 31.22, std = 5.3.

\begin{enumerate}[label=\alph*)]
    \item Using the normal distribution, compute the proportion of US countries where at least 40\% of residents have an Associate's degree.
          \begin{align*}
              P(X \geq 40)           \\
              X \sim N(31.22, 5.3^2) \\
              \mathcal{Z} = (X - \mu)/\sigma
          \end{align*}
          \begin{align*}
              P(X \geq 40)  & = P(\frac{x\mu}{\sigma} \geq \frac{40-\mu}{\sigma}) \\
              P(\mathcal{Z} & \geq\frac{40-31.22}{5.3})                           \\
                            & = 1 - P(\mathcal{Z} < 1.657)                        \\
                            & = 1 - P(\mathcal{Z} \leq 1.657)                     \\
                            & = 1 - \text{pnorm(1.657)}                           \\
                            & = 1 - 0.9512403 = 0.0487597
          \end{align*}
    \item Let \(Y\) denote per capita income represented by \(Y = 45000 + 1.7X\). \\
          \begin{align*}
              Y = aX+b, \mu = au+b, \sigma = a^2\sigma^2
          \end{align*}
          The distribution of \(Y\) is normal with mean \(\mu = 45000 + 1.7\cdot 31.22 = 45000 + 53.474 = 45053.474\) and standard deviation \(\sigma = 1.7\cdot 5.3 = 8.91\).
    \item Calculate \(c\) such that \(P(|Y-45053.07| < c) = 0.758\)
          \begin{align*}
              P(-c < Y-45053.07 < c) = 0.758                                                                         \\
              = P(-c < Y-45053.07) + P(Y-45053.07 < c)                                                               \\
              = P(-c + 45053.07 < Y < c + 45053.07)                                                                  \\
              = P(\frac{-c+45053.07-\mu_y}{\sigma_y} < \frac{Y-\mu_y}{\sigma_y} < \frac{c+45053.07-\mu_y}{\sigma_y}) \\
              = P(\frac{-c}{\sigma_y} < \mathcal{Z} < \frac{c}{\sigma_y})                                            \\
              = 1 - 2P(\mathcal{Z} < \frac{-c}{\sigma_y})                                                            \\
              P(\mathcal{Z}<\frac{-c}{\sigma_y}) = \frac{1-0.758}{2}                                                 \\
              =                                  0.121                                                               \\
              %   = 1 - P(\mathcal{Z} < \frac{-c}{\sigma_y}) - P(\mathcal{Z} > \frac{c}{\sigma_y})                       \\
          \end{align*}
          where \(\sigma_y\) is the standard deviation of Y.
\end{enumerate}

\end{document}