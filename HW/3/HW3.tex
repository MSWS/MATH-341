\documentclass{article}

\usepackage[tmargin=0.5in,bmargin=0.25in]{geometry}
\usepackage{amsmath, amssymb, amsthm}
\usepackage{enumitem}
\usepackage{tikz}

\title{\vspace{-5ex}MATH 341 HW 3}
\author{Isaac Boaz}

\begin{document}
\tikzstyle{level 1}=[level distance=3.5cm, sibling distance=3.5cm]
\tikzstyle{level 2}=[level distance=3.5cm, sibling distance=2cm]

% Define styles for bags and leafs
\tikzstyle{bag} = [text width=2em, text centered]
\tikzstyle{end} = []

\tikzstyle{edge from parent} = [draw, -latex]

\maketitle

\section*{Problem 4}
Measuring the effectiveness of treatment.
\begin{itemize}[noitemsep]
    \item \(RR\): Relative Risk
    \item \(ARR\): Absolute Risk Reduction
    \item \(T\): Treatment Group \(\implies \overline{T}\): Control Group
    \item \(B\): `Bad Outcome'
\end{itemize}


\begin{enumerate}[label=\alph*)]
    \item \(RR = P(B \vert T) / P(B \vert \overline{T})\) \\
          RR should be \(> 1\) if the treatment increases the chances of harm, and \(< 1\) if it reduces the chances of harm.
          As such, the numerator represents people that got `bad' despite being treated, and the denominator represents people that got `bad' without being treated.
    \item \(ARR = P(B \vert \overline{T}) - P(B \vert T)\) \\
          The \textbf{A} in ARR stands for \textbf{A}bsolute, and as such, it is the difference between the two probabilities.
    \item Is it possible to have a sutation where \(RR \approx 0 \text{, and} ARR \approx 0\)? \\
          Consider a ``Rare disease'', where

          Yes, it is possible to have a low relative risk (RR) and a high absolute risk reduction (ARR) when the baseline risk of the outcome is low, and the treatment has a large effect on reducing the risk.

          Example: Consider a treatment for a rare disease that affects only 1 in 10,000 people. If the treatment is given to 1000 people with the disease, it prevents one death. Without the treatment, one death would have occurred.

          \begin{align*}
              RR = \frac{(1/1000)}{1/10000} = 0.1 \\
              ARR = \frac{1}{1000} = 0.1\%
          \end{align*}

          In this example, the relative risk is low (10\% reduction) but the absolute risk reduction is high (0.1\%).

    \item Show that \(P(B \vert \overline{T}) = \frac{ARR}{1-RR}\)
          \begin{align*}
              \text{Let } a = P(B \vert \overline{T}) \text{ and } b = P(B \vert T) \\
              \frac{
                  ARR
              }{
                  1-RR
              } & = \frac{a - b}{ 1 - b/a}                                          \\
                & = \frac{a}{a} \cdot \frac{a-b}{1 - b/a}                           \\
                & = \frac{a(a-b)}{a-b}                                              \\
                & = a =P(B \vert \overline{T})
          \end{align*}
\end{enumerate}

\pagebreak

\section*{Problem 8}
The following website compares different algorithms for predicting Australian credit
approval results for 517 individuals. We focus on the confusion matrix (where each number is
divided by the total) for the logistic regression model.

\begin{tabular}{|l|l|l|l|}
    \hline
                      & Predicted Denied & Predicted Approved & Total   \\
    \hline
    Actually Denied   & 249/517          & 38/517             & 287/517 \\
    \hline
    Actually Approved & 18/517           & 149/517            & 230/517 \\
    \hline
    Total             & 330/517          & 187/517            & 1       \\
    \hline
\end{tabular}

\begin{tikzpicture}[grow=right, sloped]
    \node[bag] {\(\cdot\)}
    child {
    node[bag] {\(\overline{B}\)}
    child {
    node[
    end, label=right:
    {\(\overline{A} \vert \overline{B}\)}
    ] {}
    edge from parent
    node[above] {\(\frac{249}{517}\)}
    }
    child {
    node[end, label=right:
    {\(A \vert \overline{B}\)}
    ] {}
    edge from parent
    node[above] {\(\frac{18}{517}\)}
    }
    edge from parent
    node[above] {\(\frac{330}{517}\)}
    }
    child {
    node[bag] {\(B\)}
    child {
    node[end, label=right:
    {\(\overline{A} \vert B\)}] {}
    edge from parent
    node[above] {\(\frac{38}{517}\)}
    }
    child {
    node[end, label=right:
    {\(A \vert B\)}] {}
    edge from parent
    node[above] {\(\frac{149}{517}\)}
    }
    edge from parent
    node[above] {\(\frac{187}{517}\)}
    };
\end{tikzpicture}

\begin{enumerate}[label=\alph*)]
    \item Let A denote ‘Actually Approved’ and B denote ‘Predicted Approved’. Show
          mathematically whether or not A and B are independent
          \begin{enumerate}[label=Method \arabic*.]
              \item We know A and B are independent if \(P(A \cap B) = P(A) \cdot P(B)\).
                    \begin{align*}
                        P(A)            & = \frac{230}{517}                           \\
                        P(B)            & = \frac{187}{517}                           \\
                        P(A) \cdot P(B) & = \frac{230}{517} \cdot \frac{187}{517}     \\
                                        & = \frac{43010}{267289} = \frac{3910}{24299} \\
                        P(A \cap B)     & = \frac{149}{517}                           \\
                        \frac{149}{517} & \neq \frac{3910}{24299}                     \\
                    \end{align*}
                    Since this is not the case, we can conclude that \(A\) and \(B\) are not independent.
              \item We could also check if A is independent of B if \(P(A \vert B) = P(A)\)
                    \begin{align*}
                        P(A \vert B) & = \frac{149}{517} \cdot \frac{187}{517}     \\
                                     & = \frac{27863}{267289} \approx 0.1042429729 \\
                        P(A)         & = \frac{360}{517} \approx 0.6963249516      \\
                        P(A \vert B) & \neq P(A)
                    \end{align*}
                    Since \(P(A \vert B \neq P(A))\), we know that \(A\) is not independent of \(B\).
              \item We could also check if B is independent of A if \(P(B \vert A) = P(B)\)
                    \begin{align*}
                        P(B \vert A)           & = \frac{P(B \cap A)}{P(B)}                             \\
                                               & =\frac{P(A \cap B)}{P(A)}                              \\
                                               & = \frac{P(A \vert B) \cdot P(B)}{P(A)}                 \\
                                               & \approx \frac{0.10424 \cdot 0.361702127}{0.6963249516} \\
                                               & \approx 0.05414688879                                  \\
                        P(B) = \frac{187}{517} & \approx   0.361702127                                  \\
                        P(B \vert A)           & \neq P(B)
                    \end{align*}
                    Since \(P(B \vert A) \neq P(B)\), we know \(B\) is not independent of \(A\).
          \end{enumerate}

    \item Does the result from (a) seem to indicate that the logistic regression model has some
          ability to predict the outcome correctly? In other words, discuss what must happen if the
          logistic regression model simply generates random predictions. \\
          This model would not be able to accurately predict the outcome correctly, as the events are not independent.
    \item Let \(C\) denote ‘Actually Denied’ and \(D\) denote ‘Predicted Denied’. Without doing any
          calculation, discuss whether or not \(C\) and \(D\) are independent by referring to the result from (a)
          and one of the results mentioned in the lecture notes. \\
          We can rewrite these as
          \begin{align*}
              C & = \text{`Actually Denied'} = \overline{A}  \\
              D & = \text{`Predicted Denied'} = \overline{B} \\
          \end{align*}
          Since we know that \(A\) and \(B\) are not independent, we can conclude that \(C\) and \(D\) are also not independent.
\end{enumerate}

\pagebreak

\section*{Problem 11}
\begin{tikzpicture}[grow=right, sloped]
    \node[bag] {\(\cdot\)}
    child {
    node[bag] {\(\overline{D}\)}
    child {
    node[
    end, label=right:
    {\(\overline{N} \vert \overline{D}\)}
    ] {}
    edge from parent
    node[above] {\(0.03\)}
    }
    child {
    node[end, label=right:
    {\(N \vert \overline{D}\)}
    ] {}
    edge from parent
    node[above] {\(0.97\)}
    }
    edge from parent
    node[above] {\(0.975\)}
    }
    child {
    node[bag] {\(D\)}
    child {
    node[end, label=right:
    {\(\overline{N} \vert D\)}] {}
    edge from parent
    node[above] {\(0.80\)}
    }
    child {
    node[end, label=right:
    {\(N \vert D\)}] {}
    edge from parent
    node[above] {\(0.2\)}
    }
    edge from parent
    node[above] {\(0.025\)}
    };
\end{tikzpicture}

\begin{enumerate}[label=\alph*)]
    \item What is the probability that an individual tests positive?
          \begin{align*}
              P(\overline{N}) & = P(\overline{N} | D) + P(\overline{N} | \overline{D}) \\
                              & = 0.8 \cdot 0.025 + 0.03 \cdot 0.975                   \\
                              & = 0.04905
          \end{align*}
    \item Given an individual tests positive, what is the probability that they have the disease?
          \begin{align*}
              P(D\ \vert\ \overline{N}) & = \frac{P(D \cap \overline{N})}{P(\overline{N})}         \\
                                        & = \frac{P(\overline{N} \cap D)}{P(\overline{N})}         \\
                                        & = \frac{P(\overline{N}\ \vert\ D) P(D)}{P(\overline{N})} \\
                                        & = \frac{0.8 \cdot 0.025}{0.04905} = 0.40774719673
          \end{align*}
    \item Given an individual tests negative, what is the probability that they don't have the disease?
          \begin{align*}
              P(\overline{D} \vert N) & = \frac{P(\overline{D} \cap N)}{P(N)}                    \\
                                      & = \frac{P(N \cap \overline{D})}{P(N)}                    \\
                                      & = \frac{P(N \vert \overline{D}) P(\overline{D})}{P(N)}   \\
                                      & = \frac{0.97 \cdot 0.975}{1 - P(\overline{N})}           \\
                                      & = \frac{0.97 \cdot 0.975}{1 - 0.04905} \approx 0.9945317
          \end{align*}
    \item Comment on (b) and (c). \\
          I found these results rather surprising, as it shows that WHO cares (statistically) more about false positives than false negatives. Considering the probability of having the disease given a positive test is significantly lower than the probability of not having the disease given a negative test, it goes to show how the multiplicative aspect probabilities can heavily impact raw statistics.
\end{enumerate}

\end{document}