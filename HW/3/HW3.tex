\documentclass{article}

\usepackage[tmargin=0.5in,bmargin=0.25in]{geometry}
\usepackage{amsmath, amssymb, amsthm}
\usepackage{enumitem}

\title{\vspace{-5ex}MATH 341 HW 3}
\author{Isaac Boaz}

\begin{document}
\maketitle

\section*{Problem 4}
Measuring the effectiveness of treatment.
\begin{itemize}[noitemsep]
    \item \(RR\): Relative Risk
    \item \(ARR\): Absolute Risk Reduction
    \item \(T\): Treatment Group \(\implies \overline{T}\): Control Group
    \item \(B\): 'Bad Outcome'
\end{itemize}


\begin{enumerate}[label=\alph*)]
    \item \(RR = P(B \vert T) / P(B \vert \overline{T})\) \\
          RR should be \(> 1\) if the treatment increases the chances of harm, and \(< 1\) if it reduces the chances of harm.
          In this instance, the numerator represents the number of people harmed in total, with the treatment group
    \item \(ARR = P(B \vert \overline{T}) - P(B \vert T)\) \\
          As the paper explains how \(P(B \vert \overline{T})\)
    \item Is it possible to have a sutation where \(RR \approx 0 \text{, and} ARR \approx 0\)? \\
          Consider a "Rare disease", where
          \begin{align*}
              P(B \vert \overline{T}) & = 0.001                                  \\
              P(B \vert T)            & = \text{(smaller than above)}            \\
              ARR                     & = P(B \vert \overline{T}) - P(B \vert T)
          \end{align*}
    \item Show that \(P(B \vert \overline{T}) = \frac{ARR}{1-RR}\)
          \begin{align*}
              \frac{
                  ARR
              }{
                  1-RR
              } & =
              \frac{
                  P(B \vert \overline{T}) - P(B \vert T)
              }
              {
                  1 - \lbrack
                  \frac{
                      P (B \vert T)
                  }{
                      P(B \vert \overline{T})
                  } \rbrack
              }     \\
                & =
              \frac{
                  P(B \vert \overline{T})
              }
              {
                  P(B \vert \overline{T})
              } \cdot
              \frac{
                  P(B \vert \overline{T}) - P(B \vert \overline{T})
              }
              {
                  1 - \lbrack \frac{
                      P (B \vert T)
                  }
                  {
                      P (B \vert \overline{T})
                  } \rbrack
              }
          \end{align*}
\end{enumerate}

\pagebreak

\section*{Problem 8}
The following website compares different algorithms for predicting Australian credit
approval results for 517 individuals. We focus on the confusion matrix (where each number is
divided by the total) for the logistic regression model.

\begin{tabular}{|l|l|l|l|}
    \hline
                      & Predicted Denied & Predicted Approved & Total   \\
    \hline
    Actually Denied   & 249/517          & 38/517             & 287/517 \\
    \hline
    Actually Approved & 18/517           & 149/517            & 230/517 \\
    \hline
    Total             & 330/517          & 187/517            & 1       \\
    \hline
\end{tabular}

\begin{enumerate}[label=\alph*)]
    \item Let A denote ‘Actually Approved’ and B denote ‘Predicted Approved’. Show
          mathematically whether or not A and B are independent \\
          We know A and B are independent if \(P(A \cap B) = P(A) \cdot P(B)\).
          \begin{align*}
              P(A)            & = \frac{230}{517}                           \\
              P(B)            & = \frac{187}{517}                           \\
              P(A) \cdot P(B) & = \frac{230}{517} \cdot \frac{187}{517}     \\
                              & = \frac{43010}{267289} = \frac{3910}{24299} \\
              P(A \cap B)     & = \frac{149}{517}                           \\
              \frac{149}{517} & \neq \frac{3910}{24299}                     \\
          \end{align*}
          Since this is not the case, we can conclude that \(A\) and \(B\) are not independent.
    \item Does the result from (a) seem to indicate that the logistic regression model has some
          ability to predict the outcome correctly? In other words, discuss what must happen if the
          logistic regression model simply generates random predictions. \\
          This model would not be able to accurately predict the outcome correctly, as the events are not independent.
    \item Let \(C\) denote ‘Actually Denied’ and \(D\) denote ‘Predicted Denied’. Without doing any
          calculation, discuss whether or not \(C\) and \(D\) are independent by referring to the result from (a)
          and one of the results mentioned in the lecture notes. \\
          We can rewrite these as
          \begin{align*}
              C & = \text{`Actually Denied'} = \overline{A}  \\
              D & = \text{`Predicted Denied'} = \overline{B} \\
          \end{align*}
          Since we know that \(A\) and \(B\) are not independent, we can conclude that \(C\) and \(D\) are also not independent.
\end{enumerate}

\pagebreak

\section*{Problem 11}
\begin{align*}
    P(D) = 0.025                    \\
    P(C \vert\overline{N}) = 0.80   \\
    P(\overline{C} \vert N) = 0.97\ \\
\end{align*}

\end{document}