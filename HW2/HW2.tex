\documentclass{article}

\usepackage[tmargin=0.5in,bmargin=0.25in]{geometry}
\usepackage{amsmath, amssymb, amsthm}
\usepackage{enumitem}

\title{\vspace{-5ex}MATH 341 HW 2}
\author{Isaac Boaz}

\begin{document}

\maketitle

\section*{Problem 2}
Prove the result (3) in “MATH341 07 Chapter 1 \#7 Some Important Results Related to
Probability”. That is,
\begin{align*}
    P(A\ \cup&\ B \cup C) \\
    = P(A) + P(B) + P(C) - P(A \cap B) -& P(A \cap C) - P(B \cap C) + P(A \cap B \cap C)
\end{align*}
\begin{proof}[Proof. Additive Rule]
    \begin{align*}
        P(A \cup B) &=\ P(A) + P(B) - P(A \cap B) \\
        B \cap \overline{A} &=\ C \\
        A \cap C &=\ \varnothing \\
        P(A \cup C) &=\ P(A) + P(C) \\
        &=\ P(A) + P(B \cap \overline{A}) \\
        B &=\ (B \cap \overline{A}) \cup (A \cap B) \\
        P(B) &=\ P(B \cap \overline{A}) + P(A \cap B) \\
        P(B) - P(A \cap B) &=\ P(B \cap \overline{A}) \\
        P(A \cup B) &=\ P(A) + P(B) - P(A \cap B)
    \end{align*}
\end{proof}
Revisiting the original question:
\begin{proof}
        Let \(D = B \cup C\)
        \begin{align*}
            P(A \cup B \cup C) &= P(A \cup D) \\
            P(A \cup D) &= P(A) + P(D) - P(A \cap D) \\
            &= P(A) + P(B \cup C) - P(A \cap (B \cup C)) \\
            P(B \cup C) &= P(B) + P(C) - P(B \cap C) \\
            P(A \cup B \cup C) &= P(A) + (P(B) + P(C) - P(B \cap C)) - P(A \cap D) \\
            &= P(A) + P(B) + P(C) - P(B \cap C) - P(A \cap D) \\
            P(A \cap D) &= P(A \cap (B \cup C)) \\
            &= P((A \cap B) \cup (A \cap C)) \\
        \end{align*}
        Let \(X = (A \cap B),\ Y = (A \cap C)\)
        \begin{align*}
            P((A& \cap B) \cup (A \cap C)) \\
            &= P(X) + P(Y) - P(X \cap Y) \\
            &= P(A \cap B) + P(A \cap C) - P((A \cap B) \cap (A \cap C)) \\
            &= P(A \cap B) + P(A \cap C) - P(A \cap B \cap C) \\
            P(A& \cup B \cup C) \\
            &= P(A) + P(B) + P(C) - P(B \cap C) - P(A \cap B) - P(A \cap C) + P(A \cap B \cap C) \\
            &= P(A) + P(B) + P(C) - P(A \cap B) - P(A \cap C) - P(B \cap C) + P(A \cap B \cap C)
        \end{align*}
\end{proof}

\section*{Problem 3}
According to the survey conducted by The List, 60\% of respondents are dog lovers and 23\% of them are cat lovers. Also, 11\% of them are not fans of either pet.

\begin{enumerate}[label=\alph*)]
    \item Assuming \(P(D) = 0.6,\ P(C) = 0.23\) \\
    \begin{tabular}{|c|c|c|c|}
        \hline
        & C & \(\overline{C}\) & Total \\
        \hline
        D & -0.06 & 0.66 & 0.6 \\
        \hline
        \(\overline{D}\) & 0.29 & 0.11 & 0.4 \\
        \hline
        Total & 0.23 & 0.77 & 1 \\
        \hline
    \end{tabular} \\
    This table does not make sense, as \(C \cap D\) is \(< 0\), which is impossible under \(0 \le P(A) \le 1\).
    \item We assume that "dog lovers" and "cat lovers" are exclusive (i.e. \(D \cap C = \varnothing\)). \\
    However, this assumption is incorrect.
    \item  Assuming \(P(D) = 0.8,\ P(C) = 0.3\):\\
    \begin{tabular}{|c|c|c|c|}
        \hline
        & C & \(\overline{C}\) & Total \\
        \hline
        D & 0.14 & 0.66 & 0.8 \\
        \hline
        \(\overline{D}\) & 0.16 & 0.11 & 0.2 \\
        \hline
        Total & 0.3 & 0.7 & 1 \\
        \hline
    \end{tabular} \\
    Allows us to calculate \(P(\overline{C} \cup D)\) = 0.66. \\
    Finally, finding respondents who love cats (and not dogs) is \(P(C \cap \overline{D}) = 0.16\)
\end{enumerate}

\pagebreak

\section*{Problem 13}
\begin{enumerate}[label=\alph*)]
    \item Compute the probability of getting a “full house”.
    \begin{enumerate}[label=\arabic*.]
        \item One way of calculating this is first assigning the first pair's rank. (13 ranks, picking 1)
        \item Assign the suit of the pair. (4 suits, picking 2)
        \item Assign the rank for the last 3 cards. (12 ranks, picking 1)
        \item Lastly, these 3 cards can be of any suit (4 suits, picking 3)
        \begin{equation*}
            \begin{pmatrix}
                13 \\
                1
            \end{pmatrix}
            \begin{pmatrix}
                4 \\
                2
            \end{pmatrix}
            \cdot
            \begin{pmatrix}
                12 \\
                1
            \end{pmatrix}
            \begin{pmatrix}
                4 \\
                3
            \end{pmatrix}
        \end{equation*}
    \end{enumerate}
    \begin{enumerate}[label=\arabic*.]
        \item Alternatively, you could assign the three-card's rank first (13 ranks, picking 1)
        \item Assign the suits (4 suits, picking 3)
        \item Assign the remaining pair's rank (12 ranks, picking 1)
        \item Pick remaining suits (4 suits, picking 2)
        \begin{equation*}
            \begin{pmatrix}
                13 \\
                1
            \end{pmatrix}
            \begin{pmatrix}
                4 \\
                3
            \end{pmatrix}
            \cdot
            \begin{pmatrix}
                12 \\
                1
            \end{pmatrix}
            \begin{pmatrix}
                4 \\
                2
            \end{pmatrix}
        \end{equation*}
    \end{enumerate}
    Both methods predictably provide the same "ways of getting a full house" at \(3744\).
    Finally, plugging in the total ways one can draw any 5 cards, we get:
    \begin{equation*}
        \frac{3744}{
            \begin{pmatrix}
                52 \\
                5
            \end{pmatrix}
        } = \frac{3744}{2598960} = 0.00144
    \end{equation*}
    \item Compute the permutations of having a pair of aces and a triple of cards of another rank. \\[1\baselineskip]
    The math for this is the same as the previous problem, with the removal of picking any rank (1 rank, picking 1)
    \begin{equation*}
        \begin{pmatrix}
            1 \\
            1
        \end{pmatrix}
        \begin{pmatrix}
            4 \\
            2
        \end{pmatrix}
        \cdot
        \begin{pmatrix}
            12 \\
            1
        \end{pmatrix}
        \begin{pmatrix}
            4 \\
            3
        \end{pmatrix}
    \end{equation*}
    Giving us \(288\) ways of getting a pair of aces and a triple of cards of another rank. Doing the math:
    \begin{equation*}
        \frac{288}{
            \begin{pmatrix}
                52 \\
                5
            \end{pmatrix}
        } = \frac{288}{2598960} = 0.0001108136
        \end{equation*}
    Shows us the probability is \(< 0.02\%\) of getting two aces and triple of cards of another rank.
    Since we're asked to find the number of \textbf{permutations} possible, we can multiply\\ 
    \(288 \cdot 5!\ \text{giving us}\ 34560\) different permutations of a full house with a pair of Aces.
\end{enumerate}

\end{document}