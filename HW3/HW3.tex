\documentclass{article}

\usepackage[tmargin=0.5in,bmargin=0.25in]{geometry}
\usepackage{amsmath, amssymb, amsthm}
\usepackage{enumitem}

\title{\vspace{-5ex}MATH 341 HW 2}
\author{Isaac Boaz}

\begin{document}
\maketitle

\section*{Problem 4}
Measuring effectiveness of a treatment.
\(RR\): Relative Risk
\(ARR\): Absolute Risk Reduction

\begin{enumerate}[label=\alph*)]
    \item \(RR = P(B \vert T) / P(B \vert \overline{T})\) \\
    RR should be \(> 1\) if the treatment increases chances of harm, and \(< 1\) if it reduces the chances of harm.
    This equation makes sense, as the numerator will increase 
    \item \(ARR = P(B \vert \overline{T}) - P(B \vert T)\) \\
    As the paper explains how \(P(B \vert \overline{T})\) 
    \item Is it possible to have a sutation where \(RR \approx 0 \text{, and} ARR \approx 0\)?
    \item Show that \(P(B \vert \overline{T}) = \frac{ARR}{1-RR}\) \\
    \begin{align*}
        \frac{ARR}{1-RR} &= \frac{P(B \vert \overline{T}) - P(B \vert T)}{1 - \lbrack \frac{P (B \vert T)}{P(B \vert \overline{T})} \rbrack} \\
        &= \frac{P(B \vert \overline{T})}{P(B \vert \overline{T})} \cdot \frac{P(B \vert \overline{T}) - P(B \vert \overline{T})}{ 1 - \lbrack \frac{P (B \vert T)}{P (B \vert \overline{T})}}
    \end{align*}
\end{enumerate}

\end{document}