\documentclass{article}

\usepackage[tmargin=0.5in,bmargin=0.25in]{geometry}
\usepackage{amsmath, amssymb, amsthm}
\usepackage{enumitem}

\title{\vspace{-5ex}MATH 341 HW 3}
\author{Isaac Boaz}

\begin{document}
\maketitle

\section*{Problem 4}
Measuring effectiveness of a treatment.
\begin{itemize}[noitemsep]
    \item \(RR\): Relative Risk
    \item \(ARR\): Absolute Risk Reduction
    \item \(T\): Treatment Group \(\implies \overline{T}\): Control Group
    \item \(B\): ``Bad Outcome''
\end{itemize}


\begin{enumerate}[label=\alph*)]
    \item \(RR = P(B \vert T) / P(B \vert \overline{T})\) \\
    RR should be \(> 1\) if the treatment increases chances of harm, and \(< 1\) if it reduces the chances of harm.
    This equation makes sense, as \(B | T\) represents people that were treated and still got the bad outcome, and \(B | \overline{T}\) represents people that were not treated and got the bad outcome.
    In other words, if less people got the bad outcome with the treatment than without, then the treatment is effective (i.e. \(RR < 1\)).
    \item \(ARR = P(B \vert \overline{T}) - P(B \vert T)\) \\
    This equation is simply contrasting the probability of getting a bad outcome in the non-treated control group vs. the treated control group.
    \item Is it possible to have a situation where \(RR \approx 0 \text{, and} ARR \approx 0\)? \\
    Consider a ``Rare disease'', where
    \begin{align*}
        P(B \vert \overline{T}) &= 0.001 \\
        P(B \vert T) &< P(B \vert \overline{T}) \\
        ARR &= P(B \vert \overline{T}) - P(B \vert T)
    \end{align*}
    \item Show that \(P(B \vert \overline{T}) = \frac{ARR}{1-RR}\)
    \begin{align*}
        \frac{
            ARR
        }{
            1-RR
        } &= 
        \frac{
            P(B \vert \overline{T}) - P(B \vert T)
        }
        {
            1 - \lbrack 
                \frac{
                    P (B \vert T)
                }{
                    P(B \vert \overline{T})
                } \rbrack
        } \\
        &=
        \frac{
            P(B \vert \overline{T})
        }
        {
            P(B \vert \overline{T})
        } \cdot
        \frac{
            P(B \vert \overline{T}) - P(B \vert \overline{T})
        }
        {
            1 - \lbrack \frac{
                P (B \vert T)
                }
                {
                    P (B \vert \overline{T})
                } \rbrack
        }
    \end{align*}
\end{enumerate}

\end{document}