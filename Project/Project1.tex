\documentclass{article}

\usepackage[tmargin=0.5in,bmargin=0.25in]{geometry}
\usepackage{amsmath, amssymb, amsthm}
\usepackage{enumitem}

\title{\vspace{-4ex}Math 341 Project 1}
\author{Isaac Boaz}

\renewcommand{\arraystretch}{1.2}

\begin{document}

\maketitle

\begin{enumerate}
    \item asdf
    \item As binomial only checks two outcomes (success or failure), we can assign a safety car leading as a success and otherwise as a failure. One thing to note is that these laps aren't necessarily independent, as each lap may have an impact on the safety car's deployments further down the line.
    \item The Poisson distribution measures the \# of occurrences of an event within a fixed time/space. A requirement of this distribution is that \(n \rightarrow \infty,\ p \rightarrow 0,\ \lambda = np\). As \(n\) represents the number of laps, we can assume that it should be relatively high (given a timespan of a few seasons for example). Similarly, we can assume the safety car's deployment rate should be relatively low.
    \item The \# of safety car deployments can be modeled as a poisson distribution, and thus, the interval between each deployment can be represented by an exponential distribution.
    \item We can assume that the two time periods are independent of each other as they are disjoint.
    \item
          \begin{align*}
              P(X \geq t_1 + t_2 \mid X \geq t_1) & = P(X \geq t_2), t_1 \geq 0, t_2 \geq 0 \\
              t_1                                 & = 3                                     \\
              t_2                                 & = 5                                     \\
              P(X \geq 5 + 3 \mid X \geq 3)       & = P(X \geq 8 \mid X \geq 3)             \\
                                                  & = P(X \geq 5)
          \end{align*}

          As the memoryless' property name implies, the probability of an event occurring at a time \(t\) is independent of the time that has passed since the event occurred. Thus, the probability of an event occurring at time \(t_1 + t_2\) is independent of the probability of the event occurring at time \(t_1\). Therefore, the probability of an event occurring at time \(t_1 + t_2\) is equal to the probability of the event occurring at time \(t_2\).
\end{enumerate}

\end{document}